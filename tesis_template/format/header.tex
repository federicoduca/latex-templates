%%%%%%%% PAGE %%%%%%%%

%% PAGE: SIZE
\usepackage{geometry}
\geometry{top=3cm,bottom=3cm}

%%%%%%%% TEXT %%%%%%%%

%% TEXT: FONTS
\usepackage[english,spanish]{babel}        % set English format
% \usepackage{libertine}  % XeLaTex 'Spanish accents'
\usepackage[T1]{fontenc}  % https://tex.stackexchange.com/questions/664/why-should-i-use-usepackaget1fontenc
% \usepackage{fontspec}

%% TEXT: COLOR
\usepackage{xcolor}
\usepackage{tcolorbox}

%% TEXT: PARAGRAPH SPACES
% \setlength{\parindent}{0ex}
\setlength{\parskip}{1.5ex}
\renewcommand{\baselinestretch}{1.2}

%% TEXT: ITEMIZE
\usepackage{enumitem}        % enumerate
%    \setlist[itemize,1]{label=$\mathbf{\bullet}$}
%    \setlist[itemize,2]{label=$\mathbf{\circ}$}
%    \setlist[itemize]{noitemsep, topsep=0pt}

%% TEXT: TABLES
\usepackage{tabularx}
\usepackage{booktabs}   % nice tabs
\usepackage{multirow}   % multiple rows in tables
\usepackage{array}      % config tables format

%% TEXT: FIGURES
% \usepackage{subcaption}
%\usepackage{subfigure}
\usepackage{graphicx}
\usepackage{float}  % [H] fixed figures
\usepackage{svg}    % insert vector figures
% \usepackage{tikz}       % gráficos de tikz
% \usepackage[mode=buildnew]{standalone}

%% TEXT: LINKS
% Warning: choose wisely your package but only one
%\usepackage{varioref}
%\usepackage{cleveref}
\usepackage[linktocpage]{hyperref}
\hypersetup{
    colorlinks,
    citecolor=black,
    filecolor=black,
    linkcolor=black,
    urlcolor=black,
}

%%%%%%%% BIB %%%%%%%%
\usepackage[nonamebreak]{natbib}
\bibliographystyle{plain}
\setcitestyle{numbers,square,comma}
\usepackage{url}         % permite URL en bib
% \urlstyle{same}        % tipografía de la URL
% \usepackage{breakurl}

%%%%%%%% MATH %%%%%%%%
\usepackage{amsmath}
\usepackage{amsfonts}
\usepackage{amssymb}
\usepackage{amsthm}
\usepackage{mathtools}
\usepackage{textcomp, gensymb}  % \degree y otros símbolos
\usepackage{siunitx} % unidades en el SI
\sisetup{
    %  math-rm=\mathtt
    % binary-units = true            % Binary data is expressed in units of bits and bytes
    ,list-final-separator = { \translate{y} }
    ,list-pair-separator = { \translate{y} }
    ,range-phrase = { \translate{a} }
    ,output-decimal-marker = {,}
}

%%%%%%%% CODE %%%%%%%%
\usepackage{listings}
\colorlet{coding_color}{white}
\colorlet{number_color}{gray}
\colorlet{string_color}{orange}
\colorlet{backgr_color}{darkgray}
\colorlet{keywrd_color}{cyan}

% \lstdefinestyle{mystyle}{
%     backgroundcolor=\color{backgr_color},
%     commentstyle=\color{coding_color},
%     keywordstyle=\color{keywrd_color},
%     numberstyle=\tiny\color{number_color},
%     stringstyle=\color{string_color},
%     basicstyle=\ttfamily\footnotesize,
%     breakatwhitespace=false,
%     breaklines=true,
%     captionpos=b,
%     keepspaces=true,
%     numbers=left,
%     numbersep=5pt,
%     showspaces=false,
%     showstringspaces=false,
%     showtabs=false,
%     tabsize=2
% }
% \lstset{style=mystyle}

%%%%%%%% THESIS %%%%%%%%
\usepackage[nottoc]{tocbibind}

\newenvironment{preliminary}{
    \cleardoublepage
    \pagestyle{plain}
    \pagenumbering{roman}
    \frontmatter
}{
    \cleardoublepage
    \pagenumbering{arabic}
    \mainmatter
}

\newenvironment{postliminary}{
    \cleardoublepage%
    \pagestyle{empty}
}{
    \pagestyle{fancy}
}

\newenvironment{dedication}{
    \cleardoublepage
    \pagestyle{plain}
    \vspace*{\stretch{1}}% some space at the top
    \itshape             % the text is in italics
    \raggedleft          % flush to the right margin
    \addcontentsline{toc}{chapter}{Dedicatoria}
}{
    \par                 % end the paragraph
    \vspace{\stretch{3}} % space at bottom is three times that at the top
    \cleardoublepage
}

%% NAMES AND REFERENCES
% \renewcommand{\thefigure}{\arabic{section}.\arabic{figure}}
% \renewcommand{\thetable}{\arabic{section}.\arabic{table}}
\newcommand{\reffig}[1]{Figura~\ref{#1}}
\newcommand{\reftab}[1]{Tabla~\ref{#1}}
\renewcommand{\refeq}[1]{Ecuación~(\ref{#1})}

\addto\captionsspanish{
    \def\contentsname{Índice de Contenidos}
    \def\listfigurename{Índice de Figuras}
    \def\listtablename{Índice de Tablas}
    \def\figurename{Figura}
    \def\tablename{Tabla}
    \def\abstractname{Resumen}
    \def\abstractnameeng{Abstract}
    \def\bibname{Bibliografía}
}

